
% Default to the notebook output style

    


% Inherit from the specified cell style.




    
\documentclass[11pt]{article}

    
    
    \usepackage[T1]{fontenc}
    % Nicer default font (+ math font) than Computer Modern for most use cases
    \usepackage{mathpazo}

    % Basic figure setup, for now with no caption control since it's done
    % automatically by Pandoc (which extracts ![](path) syntax from Markdown).
    \usepackage{graphicx}
    % We will generate all images so they have a width \maxwidth. This means
    % that they will get their normal width if they fit onto the page, but
    % are scaled down if they would overflow the margins.
    \makeatletter
    \def\maxwidth{\ifdim\Gin@nat@width>\linewidth\linewidth
    \else\Gin@nat@width\fi}
    \makeatother
    \let\Oldincludegraphics\includegraphics
    % Set max figure width to be 80% of text width, for now hardcoded.
    \renewcommand{\includegraphics}[1]{\Oldincludegraphics[width=.8\maxwidth]{#1}}
    % Ensure that by default, figures have no caption (until we provide a
    % proper Figure object with a Caption API and a way to capture that
    % in the conversion process - todo).
    \usepackage{caption}
    \DeclareCaptionLabelFormat{nolabel}{}
    \captionsetup{labelformat=nolabel}

    \usepackage{adjustbox} % Used to constrain images to a maximum size 
    \usepackage{xcolor} % Allow colors to be defined
    \usepackage{enumerate} % Needed for markdown enumerations to work
    \usepackage{geometry} % Used to adjust the document margins
    \usepackage{amsmath} % Equations
    \usepackage{amssymb} % Equations
    \usepackage{textcomp} % defines textquotesingle
    % Hack from http://tex.stackexchange.com/a/47451/13684:
    \AtBeginDocument{%
        \def\PYZsq{\textquotesingle}% Upright quotes in Pygmentized code
    }
    \usepackage{upquote} % Upright quotes for verbatim code
    \usepackage{eurosym} % defines \euro
    \usepackage[mathletters]{ucs} % Extended unicode (utf-8) support
    \usepackage[utf8x]{inputenc} % Allow utf-8 characters in the tex document
    \usepackage{fancyvrb} % verbatim replacement that allows latex
    \usepackage{grffile} % extends the file name processing of package graphics 
                         % to support a larger range 
    % The hyperref package gives us a pdf with properly built
    % internal navigation ('pdf bookmarks' for the table of contents,
    % internal cross-reference links, web links for URLs, etc.)
    \usepackage{hyperref}
    \usepackage{longtable} % longtable support required by pandoc >1.10
    \usepackage{booktabs}  % table support for pandoc > 1.12.2
    \usepackage[inline]{enumitem} % IRkernel/repr support (it uses the enumerate* environment)
    \usepackage[normalem]{ulem} % ulem is needed to support strikethroughs (\sout)
                                % normalem makes italics be italics, not underlines
    

    
    
    % Colors for the hyperref package
    \definecolor{urlcolor}{rgb}{0,.145,.698}
    \definecolor{linkcolor}{rgb}{.71,0.21,0.01}
    \definecolor{citecolor}{rgb}{.12,.54,.11}

    % ANSI colors
    \definecolor{ansi-black}{HTML}{3E424D}
    \definecolor{ansi-black-intense}{HTML}{282C36}
    \definecolor{ansi-red}{HTML}{E75C58}
    \definecolor{ansi-red-intense}{HTML}{B22B31}
    \definecolor{ansi-green}{HTML}{00A250}
    \definecolor{ansi-green-intense}{HTML}{007427}
    \definecolor{ansi-yellow}{HTML}{DDB62B}
    \definecolor{ansi-yellow-intense}{HTML}{B27D12}
    \definecolor{ansi-blue}{HTML}{208FFB}
    \definecolor{ansi-blue-intense}{HTML}{0065CA}
    \definecolor{ansi-magenta}{HTML}{D160C4}
    \definecolor{ansi-magenta-intense}{HTML}{A03196}
    \definecolor{ansi-cyan}{HTML}{60C6C8}
    \definecolor{ansi-cyan-intense}{HTML}{258F8F}
    \definecolor{ansi-white}{HTML}{C5C1B4}
    \definecolor{ansi-white-intense}{HTML}{A1A6B2}

    % commands and environments needed by pandoc snippets
    % extracted from the output of `pandoc -s`
    \providecommand{\tightlist}{%
      \setlength{\itemsep}{0pt}\setlength{\parskip}{0pt}}
    \DefineVerbatimEnvironment{Highlighting}{Verbatim}{commandchars=\\\{\}}
    % Add ',fontsize=\small' for more characters per line
    \newenvironment{Shaded}{}{}
    \newcommand{\KeywordTok}[1]{\textcolor[rgb]{0.00,0.44,0.13}{\textbf{{#1}}}}
    \newcommand{\DataTypeTok}[1]{\textcolor[rgb]{0.56,0.13,0.00}{{#1}}}
    \newcommand{\DecValTok}[1]{\textcolor[rgb]{0.25,0.63,0.44}{{#1}}}
    \newcommand{\BaseNTok}[1]{\textcolor[rgb]{0.25,0.63,0.44}{{#1}}}
    \newcommand{\FloatTok}[1]{\textcolor[rgb]{0.25,0.63,0.44}{{#1}}}
    \newcommand{\CharTok}[1]{\textcolor[rgb]{0.25,0.44,0.63}{{#1}}}
    \newcommand{\StringTok}[1]{\textcolor[rgb]{0.25,0.44,0.63}{{#1}}}
    \newcommand{\CommentTok}[1]{\textcolor[rgb]{0.38,0.63,0.69}{\textit{{#1}}}}
    \newcommand{\OtherTok}[1]{\textcolor[rgb]{0.00,0.44,0.13}{{#1}}}
    \newcommand{\AlertTok}[1]{\textcolor[rgb]{1.00,0.00,0.00}{\textbf{{#1}}}}
    \newcommand{\FunctionTok}[1]{\textcolor[rgb]{0.02,0.16,0.49}{{#1}}}
    \newcommand{\RegionMarkerTok}[1]{{#1}}
    \newcommand{\ErrorTok}[1]{\textcolor[rgb]{1.00,0.00,0.00}{\textbf{{#1}}}}
    \newcommand{\NormalTok}[1]{{#1}}
    
    % Additional commands for more recent versions of Pandoc
    \newcommand{\ConstantTok}[1]{\textcolor[rgb]{0.53,0.00,0.00}{{#1}}}
    \newcommand{\SpecialCharTok}[1]{\textcolor[rgb]{0.25,0.44,0.63}{{#1}}}
    \newcommand{\VerbatimStringTok}[1]{\textcolor[rgb]{0.25,0.44,0.63}{{#1}}}
    \newcommand{\SpecialStringTok}[1]{\textcolor[rgb]{0.73,0.40,0.53}{{#1}}}
    \newcommand{\ImportTok}[1]{{#1}}
    \newcommand{\DocumentationTok}[1]{\textcolor[rgb]{0.73,0.13,0.13}{\textit{{#1}}}}
    \newcommand{\AnnotationTok}[1]{\textcolor[rgb]{0.38,0.63,0.69}{\textbf{\textit{{#1}}}}}
    \newcommand{\CommentVarTok}[1]{\textcolor[rgb]{0.38,0.63,0.69}{\textbf{\textit{{#1}}}}}
    \newcommand{\VariableTok}[1]{\textcolor[rgb]{0.10,0.09,0.49}{{#1}}}
    \newcommand{\ControlFlowTok}[1]{\textcolor[rgb]{0.00,0.44,0.13}{\textbf{{#1}}}}
    \newcommand{\OperatorTok}[1]{\textcolor[rgb]{0.40,0.40,0.40}{{#1}}}
    \newcommand{\BuiltInTok}[1]{{#1}}
    \newcommand{\ExtensionTok}[1]{{#1}}
    \newcommand{\PreprocessorTok}[1]{\textcolor[rgb]{0.74,0.48,0.00}{{#1}}}
    \newcommand{\AttributeTok}[1]{\textcolor[rgb]{0.49,0.56,0.16}{{#1}}}
    \newcommand{\InformationTok}[1]{\textcolor[rgb]{0.38,0.63,0.69}{\textbf{\textit{{#1}}}}}
    \newcommand{\WarningTok}[1]{\textcolor[rgb]{0.38,0.63,0.69}{\textbf{\textit{{#1}}}}}
    
    
    % Define a nice break command that doesn't care if a line doesn't already
    % exist.
    \def\br{\hspace*{\fill} \\* }
    % Math Jax compatability definitions
    \def\gt{>}
    \def\lt{<}
    % Document parameters
    \title{Chapitre 6.1 - Pr?parer son compte-rendu}
    
    
    

    % Pygments definitions
    
\makeatletter
\def\PY@reset{\let\PY@it=\relax \let\PY@bf=\relax%
    \let\PY@ul=\relax \let\PY@tc=\relax%
    \let\PY@bc=\relax \let\PY@ff=\relax}
\def\PY@tok#1{\csname PY@tok@#1\endcsname}
\def\PY@toks#1+{\ifx\relax#1\empty\else%
    \PY@tok{#1}\expandafter\PY@toks\fi}
\def\PY@do#1{\PY@bc{\PY@tc{\PY@ul{%
    \PY@it{\PY@bf{\PY@ff{#1}}}}}}}
\def\PY#1#2{\PY@reset\PY@toks#1+\relax+\PY@do{#2}}

\expandafter\def\csname PY@tok@w\endcsname{\def\PY@tc##1{\textcolor[rgb]{0.73,0.73,0.73}{##1}}}
\expandafter\def\csname PY@tok@c\endcsname{\let\PY@it=\textit\def\PY@tc##1{\textcolor[rgb]{0.25,0.50,0.50}{##1}}}
\expandafter\def\csname PY@tok@cp\endcsname{\def\PY@tc##1{\textcolor[rgb]{0.74,0.48,0.00}{##1}}}
\expandafter\def\csname PY@tok@k\endcsname{\let\PY@bf=\textbf\def\PY@tc##1{\textcolor[rgb]{0.00,0.50,0.00}{##1}}}
\expandafter\def\csname PY@tok@kp\endcsname{\def\PY@tc##1{\textcolor[rgb]{0.00,0.50,0.00}{##1}}}
\expandafter\def\csname PY@tok@kt\endcsname{\def\PY@tc##1{\textcolor[rgb]{0.69,0.00,0.25}{##1}}}
\expandafter\def\csname PY@tok@o\endcsname{\def\PY@tc##1{\textcolor[rgb]{0.40,0.40,0.40}{##1}}}
\expandafter\def\csname PY@tok@ow\endcsname{\let\PY@bf=\textbf\def\PY@tc##1{\textcolor[rgb]{0.67,0.13,1.00}{##1}}}
\expandafter\def\csname PY@tok@nb\endcsname{\def\PY@tc##1{\textcolor[rgb]{0.00,0.50,0.00}{##1}}}
\expandafter\def\csname PY@tok@nf\endcsname{\def\PY@tc##1{\textcolor[rgb]{0.00,0.00,1.00}{##1}}}
\expandafter\def\csname PY@tok@nc\endcsname{\let\PY@bf=\textbf\def\PY@tc##1{\textcolor[rgb]{0.00,0.00,1.00}{##1}}}
\expandafter\def\csname PY@tok@nn\endcsname{\let\PY@bf=\textbf\def\PY@tc##1{\textcolor[rgb]{0.00,0.00,1.00}{##1}}}
\expandafter\def\csname PY@tok@ne\endcsname{\let\PY@bf=\textbf\def\PY@tc##1{\textcolor[rgb]{0.82,0.25,0.23}{##1}}}
\expandafter\def\csname PY@tok@nv\endcsname{\def\PY@tc##1{\textcolor[rgb]{0.10,0.09,0.49}{##1}}}
\expandafter\def\csname PY@tok@no\endcsname{\def\PY@tc##1{\textcolor[rgb]{0.53,0.00,0.00}{##1}}}
\expandafter\def\csname PY@tok@nl\endcsname{\def\PY@tc##1{\textcolor[rgb]{0.63,0.63,0.00}{##1}}}
\expandafter\def\csname PY@tok@ni\endcsname{\let\PY@bf=\textbf\def\PY@tc##1{\textcolor[rgb]{0.60,0.60,0.60}{##1}}}
\expandafter\def\csname PY@tok@na\endcsname{\def\PY@tc##1{\textcolor[rgb]{0.49,0.56,0.16}{##1}}}
\expandafter\def\csname PY@tok@nt\endcsname{\let\PY@bf=\textbf\def\PY@tc##1{\textcolor[rgb]{0.00,0.50,0.00}{##1}}}
\expandafter\def\csname PY@tok@nd\endcsname{\def\PY@tc##1{\textcolor[rgb]{0.67,0.13,1.00}{##1}}}
\expandafter\def\csname PY@tok@s\endcsname{\def\PY@tc##1{\textcolor[rgb]{0.73,0.13,0.13}{##1}}}
\expandafter\def\csname PY@tok@sd\endcsname{\let\PY@it=\textit\def\PY@tc##1{\textcolor[rgb]{0.73,0.13,0.13}{##1}}}
\expandafter\def\csname PY@tok@si\endcsname{\let\PY@bf=\textbf\def\PY@tc##1{\textcolor[rgb]{0.73,0.40,0.53}{##1}}}
\expandafter\def\csname PY@tok@se\endcsname{\let\PY@bf=\textbf\def\PY@tc##1{\textcolor[rgb]{0.73,0.40,0.13}{##1}}}
\expandafter\def\csname PY@tok@sr\endcsname{\def\PY@tc##1{\textcolor[rgb]{0.73,0.40,0.53}{##1}}}
\expandafter\def\csname PY@tok@ss\endcsname{\def\PY@tc##1{\textcolor[rgb]{0.10,0.09,0.49}{##1}}}
\expandafter\def\csname PY@tok@sx\endcsname{\def\PY@tc##1{\textcolor[rgb]{0.00,0.50,0.00}{##1}}}
\expandafter\def\csname PY@tok@m\endcsname{\def\PY@tc##1{\textcolor[rgb]{0.40,0.40,0.40}{##1}}}
\expandafter\def\csname PY@tok@gh\endcsname{\let\PY@bf=\textbf\def\PY@tc##1{\textcolor[rgb]{0.00,0.00,0.50}{##1}}}
\expandafter\def\csname PY@tok@gu\endcsname{\let\PY@bf=\textbf\def\PY@tc##1{\textcolor[rgb]{0.50,0.00,0.50}{##1}}}
\expandafter\def\csname PY@tok@gd\endcsname{\def\PY@tc##1{\textcolor[rgb]{0.63,0.00,0.00}{##1}}}
\expandafter\def\csname PY@tok@gi\endcsname{\def\PY@tc##1{\textcolor[rgb]{0.00,0.63,0.00}{##1}}}
\expandafter\def\csname PY@tok@gr\endcsname{\def\PY@tc##1{\textcolor[rgb]{1.00,0.00,0.00}{##1}}}
\expandafter\def\csname PY@tok@ge\endcsname{\let\PY@it=\textit}
\expandafter\def\csname PY@tok@gs\endcsname{\let\PY@bf=\textbf}
\expandafter\def\csname PY@tok@gp\endcsname{\let\PY@bf=\textbf\def\PY@tc##1{\textcolor[rgb]{0.00,0.00,0.50}{##1}}}
\expandafter\def\csname PY@tok@go\endcsname{\def\PY@tc##1{\textcolor[rgb]{0.53,0.53,0.53}{##1}}}
\expandafter\def\csname PY@tok@gt\endcsname{\def\PY@tc##1{\textcolor[rgb]{0.00,0.27,0.87}{##1}}}
\expandafter\def\csname PY@tok@err\endcsname{\def\PY@bc##1{\setlength{\fboxsep}{0pt}\fcolorbox[rgb]{1.00,0.00,0.00}{1,1,1}{\strut ##1}}}
\expandafter\def\csname PY@tok@kc\endcsname{\let\PY@bf=\textbf\def\PY@tc##1{\textcolor[rgb]{0.00,0.50,0.00}{##1}}}
\expandafter\def\csname PY@tok@kd\endcsname{\let\PY@bf=\textbf\def\PY@tc##1{\textcolor[rgb]{0.00,0.50,0.00}{##1}}}
\expandafter\def\csname PY@tok@kn\endcsname{\let\PY@bf=\textbf\def\PY@tc##1{\textcolor[rgb]{0.00,0.50,0.00}{##1}}}
\expandafter\def\csname PY@tok@kr\endcsname{\let\PY@bf=\textbf\def\PY@tc##1{\textcolor[rgb]{0.00,0.50,0.00}{##1}}}
\expandafter\def\csname PY@tok@bp\endcsname{\def\PY@tc##1{\textcolor[rgb]{0.00,0.50,0.00}{##1}}}
\expandafter\def\csname PY@tok@fm\endcsname{\def\PY@tc##1{\textcolor[rgb]{0.00,0.00,1.00}{##1}}}
\expandafter\def\csname PY@tok@vc\endcsname{\def\PY@tc##1{\textcolor[rgb]{0.10,0.09,0.49}{##1}}}
\expandafter\def\csname PY@tok@vg\endcsname{\def\PY@tc##1{\textcolor[rgb]{0.10,0.09,0.49}{##1}}}
\expandafter\def\csname PY@tok@vi\endcsname{\def\PY@tc##1{\textcolor[rgb]{0.10,0.09,0.49}{##1}}}
\expandafter\def\csname PY@tok@vm\endcsname{\def\PY@tc##1{\textcolor[rgb]{0.10,0.09,0.49}{##1}}}
\expandafter\def\csname PY@tok@sa\endcsname{\def\PY@tc##1{\textcolor[rgb]{0.73,0.13,0.13}{##1}}}
\expandafter\def\csname PY@tok@sb\endcsname{\def\PY@tc##1{\textcolor[rgb]{0.73,0.13,0.13}{##1}}}
\expandafter\def\csname PY@tok@sc\endcsname{\def\PY@tc##1{\textcolor[rgb]{0.73,0.13,0.13}{##1}}}
\expandafter\def\csname PY@tok@dl\endcsname{\def\PY@tc##1{\textcolor[rgb]{0.73,0.13,0.13}{##1}}}
\expandafter\def\csname PY@tok@s2\endcsname{\def\PY@tc##1{\textcolor[rgb]{0.73,0.13,0.13}{##1}}}
\expandafter\def\csname PY@tok@sh\endcsname{\def\PY@tc##1{\textcolor[rgb]{0.73,0.13,0.13}{##1}}}
\expandafter\def\csname PY@tok@s1\endcsname{\def\PY@tc##1{\textcolor[rgb]{0.73,0.13,0.13}{##1}}}
\expandafter\def\csname PY@tok@mb\endcsname{\def\PY@tc##1{\textcolor[rgb]{0.40,0.40,0.40}{##1}}}
\expandafter\def\csname PY@tok@mf\endcsname{\def\PY@tc##1{\textcolor[rgb]{0.40,0.40,0.40}{##1}}}
\expandafter\def\csname PY@tok@mh\endcsname{\def\PY@tc##1{\textcolor[rgb]{0.40,0.40,0.40}{##1}}}
\expandafter\def\csname PY@tok@mi\endcsname{\def\PY@tc##1{\textcolor[rgb]{0.40,0.40,0.40}{##1}}}
\expandafter\def\csname PY@tok@il\endcsname{\def\PY@tc##1{\textcolor[rgb]{0.40,0.40,0.40}{##1}}}
\expandafter\def\csname PY@tok@mo\endcsname{\def\PY@tc##1{\textcolor[rgb]{0.40,0.40,0.40}{##1}}}
\expandafter\def\csname PY@tok@ch\endcsname{\let\PY@it=\textit\def\PY@tc##1{\textcolor[rgb]{0.25,0.50,0.50}{##1}}}
\expandafter\def\csname PY@tok@cm\endcsname{\let\PY@it=\textit\def\PY@tc##1{\textcolor[rgb]{0.25,0.50,0.50}{##1}}}
\expandafter\def\csname PY@tok@cpf\endcsname{\let\PY@it=\textit\def\PY@tc##1{\textcolor[rgb]{0.25,0.50,0.50}{##1}}}
\expandafter\def\csname PY@tok@c1\endcsname{\let\PY@it=\textit\def\PY@tc##1{\textcolor[rgb]{0.25,0.50,0.50}{##1}}}
\expandafter\def\csname PY@tok@cs\endcsname{\let\PY@it=\textit\def\PY@tc##1{\textcolor[rgb]{0.25,0.50,0.50}{##1}}}

\def\PYZbs{\char`\\}
\def\PYZus{\char`\_}
\def\PYZob{\char`\{}
\def\PYZcb{\char`\}}
\def\PYZca{\char`\^}
\def\PYZam{\char`\&}
\def\PYZlt{\char`\<}
\def\PYZgt{\char`\>}
\def\PYZsh{\char`\#}
\def\PYZpc{\char`\%}
\def\PYZdl{\char`\$}
\def\PYZhy{\char`\-}
\def\PYZsq{\char`\'}
\def\PYZdq{\char`\"}
\def\PYZti{\char`\~}
% for compatibility with earlier versions
\def\PYZat{@}
\def\PYZlb{[}
\def\PYZrb{]}
\makeatother


    % Exact colors from NB
    \definecolor{incolor}{rgb}{0.0, 0.0, 0.5}
    \definecolor{outcolor}{rgb}{0.545, 0.0, 0.0}



    
    % Prevent overflowing lines due to hard-to-break entities
    \sloppy 
    % Setup hyperref package
    \hypersetup{
      breaklinks=true,  % so long urls are correctly broken across lines
      colorlinks=true,
      urlcolor=urlcolor,
      linkcolor=linkcolor,
      citecolor=citecolor,
      }
    % Slightly bigger margins than the latex defaults
    
    \geometry{verbose,tmargin=1in,bmargin=1in,lmargin=1in,rmargin=1in}
    
    

    \begin{document}
    
    
    \maketitle
    
    

    
    \section{Chapitre 6.1 - Exigence sur le compte
rendu}\label{chapitre-6.1---exigence-sur-le-compte-rendu}

Le compte rendu est un document qui a plusieurs rôles :

\begin{itemize}
\tightlist
\item
  Évaluer votre investissement lors des séances de travaux Pratiques;
\item
  Servir de support de révisions pour les examens;
\item
  Renforcer l'apprentissage en couplant notions théoriques et pratiques;
\item
  S'entraîner pour la rédaction de document scientifique.
\end{itemize}

Les séances de travaux pratiques sont des outils que l'enseignant peut
utiliser pour montrer des applications concrètes ou bien des travaux qui
sont en cours de recherche. En fonction de votre niveau d'études et du
cours, les séances peuvent présenter des techniques mathématiques qui
sont soit spécifiques à une situation (méthode de résolution de problème
inverse par exemple) ou qui peuvent s'utiliser dans des cas plus
généraux (méthode de Newton, factorisation LU,...).

Vous devez être capable de différencier les cas, car le compte rendu
n'aura pas la même structure et l'enseignant n'aura pas les mêmes
attentes. Lorsqu'un compte rendu ne contient pas tous les éléments
nécessaires, il devient difficile de donner une note à la hauteur du
travail fourni. L'autre point négatif survient lorsque vous souhaitez
utiliser la méthode dans un autre cours ou dans la vie réelle, mais que
vous ne pouvez pas utiliser un travail complet pour vous aider.

Nous attendons pour chaque compte rendu des cours CMA1-CMA2 :

\begin{itemize}
\tightlist
\item
  L'ensemble des fichiers Matlab commentés dans un fichier compressé
  (.zip ou .rar);
\item
  Un document au format pdf.
\end{itemize}

Dans le cas de la séance d'introduction à Matlab, le compte rendu vous
sert pour noter les notions que vous jugez importantes à garder sous la
main et comme support pour les commentaires que vous avez sur la séance.

Dans ce chapitre, nous vous donnerons les directives pour les différents
éléments du rendu.

    \subsection{6.1.1 Le compte rendu pdf}\label{le-compte-rendu-pdf}

Dans la majorité des cas, il vous suffira de suivre le sujet pour avoir
une structure valable de votre compte rendu. Nous vous recommandons donc
de lire le sujet au complet avant de commencer, il est également
recommandé de lire le sujet avant la séance, vous pourrez trouver le
sujet sur le site principal.

Lors de votre lecture, essayez d'identifier :

\begin{itemize}
\tightlist
\item
  La problématique;
\item
  Les algorithmes que vous aurez à implémenter;
\item
  Les parties qui composeront votre document.
\end{itemize}

\subsubsection{Les différentes parties que l'on doit retrouver dans
votre
document}\label{les-diffuxe9rentes-parties-que-lon-doit-retrouver-dans-votre-document}

Le document doit être organisé avec :

\begin{itemize}
\tightlist
\item
  Une page de garde;
\item
  Une introduction;
\item
  Un développement;
\item
  Une conclusion;
\item
  (facultatif) Une bibliographie, des annexes, des listes de figure,
  ....
\end{itemize}

\paragraph{La page de garde}\label{la-page-de-garde}

La page de garde de votre compte rendu doit contenir :

\begin{itemize}
\tightlist
\item
  Les logos des écoles qui vous encadre (\emph{ESIR, Université de
  Rennes 1 , ISTIC});
\item
  Les noms et prénoms des membres composant votre groupe;
\item
  Le numéro de votre groupe;
\item
  Le titre de l'unité d'étude (\emph{CMA1,...});
\item
  Le titre de votre promotion ainsi que son année (\emph{cupge
  2018-2019});
\item
  Le titre du TP;
\end{itemize}

À l'arrière de votre page de garde, vous devrez rajouter la mention
suivante : \textbf{\emph{"J'atteste que ce travail est original, qu'il
indique de façon appropriée tous les emprunts, et qu'il fait référence
de façon appropriée à chaque source utilisée"}}. Il s'agit d'une règle
imposée par l'ESIR à partir de l apremière année d'école d'ingénieur,
autant commencer à prendre cette habitude.

\paragraph{L'introduction}\label{lintroduction}

Vous ne devez pas commencer sa rédaction tant que vos expériences ne
sont pas terminées. C'est un contrat rédactionnel que vous avez passé
avec votre destinataire. Elle va donc aborder plusieurs sous-parties :

\begin{itemize}
\tightlist
\item
  L'annonce du thème et de la problématique
\item
  L'annonce du plan (grande partie)
\item
  Justifier la méthodologie
\end{itemize}

\paragraph{Le développement}\label{le-duxe9veloppement}

Le développement doit être organisé en sections qui doivent servir à
répondre à la problématique. Les sections sont faciles à identifier
grâce au sujet, on peut notamment avoir l'organisation :

\begin{itemize}
\tightlist
\item
  Méthologie / Résultats : La partie 'Méthologie' présente les
  différents algorithmes implémentés et la partie 'Résultats' contient
  les figures commentées des différentes expériences.
\item
  Titre correspondant à une expérience : Parfois la méthodologie est
  trop faible pour faire toute une partie, vous pouvez donc coupler une
  partie 'Méthologie / Résultats' dans une même section. Les titres des
  sections peuvent correspondre avec ceux donné en TP.
\end{itemize}

Vous pouvez bien rajouter des sous-section, des sous sous-section, ...
faites en sorte que les titres que vous donnez aient une certaine
logique.

Dans le cadre académique et universitaire, on applique une
classification décimale, c'est à dire :

\begin{itemize}
\tightlist
\item
  Partie 1 - Titre

  \begin{itemize}
  \tightlist
  \item
    1.1 - Sous-titre
  \item
    1.2 - Sous-titre

    \begin{itemize}
    \tightlist
    \item
      1.2.1 - Sous division titre
    \item
      1.2.2 - Sous division titre
    \end{itemize}
  \end{itemize}
\item
  Partie 2 - Titre
\end{itemize}

Vous devez éviter les chiffres romains qui peuvent entrainer des
confusions. On distingue plusieurs types de titres :

\begin{itemize}
\item
  Titre du compte rendu : Il est normalement sur votre première page en
  taille 16 au minimum en gras.
\item
  Le sous-titre du compte rendu : C'est le titre secondaire, placé en
  complément du titre de rubrique. Il s'agit de votre problématique qui
  doit également apparaître sur votre première page en taille 14 et en
  gras.
\item
  Les titres de section : Marque les grandes parties de votre dossier,
  comme l'introduction/partie de développement/conclusion. Taille 14 et
  en gras.
\item
  Les autres titres : Ce sont les titres secondaires qui subdivisent les
  sections. Ils sont en taille 12 et en gras.
\end{itemize}

\paragraph{La conclusion}\label{la-conclusion}

La conclusion ne doit surtout pas être bâclée, elle doit être soignée,
très rigoureuse car c'est la dernière impression que laissez à votre
lecteur. Elle se rédige en plusieurs sous parties :

\begin{itemize}
\item
  Phase de bilan ou de synthèse :

  \begin{itemize}
  \tightlist
  \item
    Synthèse des résultats, des expériences;
  \item
    Synthèse de vos analyses;
  \end{itemize}
\item
  L'ouverture :

  \begin{itemize}
  \tightlist
  \item
    Élargir votre vision du sujet;
  \end{itemize}
\end{itemize}

    \subsubsection{Le corps de texte}\label{le-corps-de-texte}

Le corps de texte doit être rédigé avec une police avec empattements
(Time New Roman par exemple) de taille 10. Les paragraphes doivent être
en format \textbf{justifiés} et commencer par une tabulation.

    \subsubsection{Les formules
mathématiques}\label{les-formules-mathuxe9matiques}

Les formules mathématiques doivent être :

\begin{itemize}
\tightlist
\item
  Soit tapées et intégrées directement dans votre document;
\item
  Soit prises en photo par capture, mais :

  \begin{itemize}
  \tightlist
  \item
    Doivent ne pas avoir de fond (images modifiées sous GIMP par exemple
    \href{https://www.gimp.org/}{lien vers le site});
  \item
    Doivent avoir une police de la même taille que le texte;
  \item
    Doivent être visibles et propres;
  \end{itemize}
\end{itemize}

Au bout de de l'équation, vous rajouterez un chiffre qui permettra d'y
faire référence dans vos explications. Voici un exemple :

\emph{Dans vos explications} : L'équation (1) représente un cas simple
de signal sur laquelle nous allons travailler, ...

    \subsubsection{Présenter ses
résultats}\label{pruxe9senter-ses-ruxe9sultats}

Une fois que vous avez présenté et implémenté les algorithmes, vous
devez résumer vos résultats dans un format adéquat (tableau de valeurs,
figure, diagramme ...). Vous devez ajouter cette représentation dans
votre compte rendu, pour ensuite la commenter et conclure sur
l'expérience que vous avez réalisé. Vous n'oublierez pas d'ajouter une
légende à cet élément ("Tableau 1 -", "Figure 1 -", ...), afin d'y faire
référence dans vos observations

\paragraph{Les figures}\label{les-figures}

L'ensemble des figures doivent avoir :

\begin{itemize}
\tightlist
\item
  Un titre;
\item
  Des titres d'axe;
\item
  Une légende si vous représentez plusieurs choses;
\end{itemize}

Voici un exemple que vous pouvez utiliser :

Faites attention, les figures représente vos résultats, elles doivent
donc être belles et lisibles. Ainsi, vous ne devez pas vous contenter de
faire une capture d'écran bâclée, vous devez les exporter. Sous Matlab,
il vous suffit de cliquer sur l'icône de disquette pour enregistrer
votre figure, mais attention le format par défaut est en \emph{figure
Matlab (.fig)}. Vous devez donc sélectionner un autre format avant de
sauvegarder, sinon vous ne serez plus capable d'ouvrir vos figures sans
Matlab. Nous vous recommandons le format \emph{png} ou \emph{svg} pour
vos figures.

\paragraph{Les tableaux}\label{les-tableaux}

Les tableaux doivent également être présenté comme il se doit :

    \subsubsection{Présenter ses
algorithmes}\label{pruxe9senter-ses-algorithmes}

Nous pouvons vous demander de rédiger les algorithmes par vous-même à
partir des connaissances acquises en TD. Sachant qu'il s'agit d'un
travail de méthodologie, vous devez l'inclure dans votre rapport. Il
existe un format pour ça : le pseudo-code
\href{https://en.wikipedia.org/wiki/Pseudocode}{wiki}.

Le meilleur moyen de rédiger un algorithme est de passer par LateX
\href{https://www.latex-project.org/}{ref}, cependant vous n'êtes pas
encore familiarisé avec cet outil. Vous allez devoir trouver une astuce
pour faire ressortir vos pseudo-codes comme :

\begin{itemize}
\tightlist
\item
  Changer de police;
\item
  Centrer le texte;
\item
  ...
\end{itemize}

Comme pour les figures et les tableaux, vous devez lui ajouter une
légende, voici un exemple :

    \subsection{6.1.2 Présentation du code}\label{pruxe9sentation-du-code}

\subsubsection{Les noms de variables}\label{les-noms-de-variables}

Les noms de variables, de scripts et de fonctions doivent avoir du sens
au maximum. Ainsi, nous pourrons accepter des variables telles que 'i',
'j' pour des index dans les boucles, mais pas pour les variables
importantes.

\textbf{NB :} Les variables que l'on utilise dans les formules
mathématiques comme 'S', 'X', ... peuvent être considérés comme ayant du
sens.

\subsubsection{Organisation des
fichiers}\label{organisation-des-fichiers}

Lorsque l'on demande d'implémenter des fonctions, il serait bon d'avoir
un script qui l'appelle et qui ressort les observations faites. Ainsi,
vous devrez rendre les fichiers '.m' contenant vos algorithmes et un
script principal qui effectue toutes vos expériences. Afin de gagner des
points supplémentaires, vous pouvez rendre votre script principal
intéractif.

\textbf{Figure 2} - Représentation de votre rendu

    \subsection{6.1.3 Conclusion}\label{conclusion}

L'ensemble des éléments vus ci-dessus compteront dans l'évaluation de
vos compte rendus. Chacune de nos grilles de notation contiennent les
mentions suivantes :

Ce bloc compte pour 1/3 de la note, donc ne le négligez pas.


    % Add a bibliography block to the postdoc
    
    
    
    \end{document}
